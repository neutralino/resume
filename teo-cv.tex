\documentclass[margin,line]{res}
\usepackage{pifont}
\usepackage{marvosym}
\usepackage{xcolor}

\oddsidemargin -.5in
\evensidemargin -.5in
\textwidth=6.0in
\itemsep=0in
\parsep=0in

\newenvironment{list1}{
  \begin{list}{\ding{113}}{%
      \setlength{\itemsep}{0in}
      \setlength{\parsep}{0in} \setlength{\parskip}{0in}
      \setlength{\topsep}{0in} \setlength{\partopsep}{0in}
      \setlength{\leftmargin}{0.17in}}}{\end{list}}
\newenvironment{list2}{
  \begin{list}{$\bullet$}{%
      \setlength{\itemsep}{0in}
      \setlength{\parsep}{0in} \setlength{\parskip}{0in}
      \setlength{\topsep}{0in} \setlength{\partopsep}{0in}
      \setlength{\leftmargin}{0.2in}}}{\end{list}}


\begin{document}


%\name{Wee Don Teo --- Curriculum Vitae\vspace*{.1in}} %\hfill \vspace{-0.15in} \Letter \hspace{1 mm} don.teo@gmail.com
\name{Wee Don Teo --- \Resume\vspace*{.1in}} %\hfill \vspace{-0.15in} \Letter \hspace{1 mm} don.teo@gmail.com

%\hfill { }

\begin{resume}
\section{\sc Contact Information}
\vspace{.05in}
\begin{tabular}{@{}p{2in}p{3.8in}}
%{\bf Current Address}          & \hfill {\bf Permanent Address}\\
%210 Summerhill Dr., Apt. 1     & \hfill 33 Terraview Blvd.\\
%Ithaca, NY 14850 USA           & \hfill Toronto, ON, M1R 4L8, Canada \\
%\Mobilefone \hspace{1 mm} 607-220-4408           & \hfill \Telefon \hspace{1 mm} 416-383-1568 \\
%                               & \hfill {\bf After March 8}\\
%%%
%324 Physical Sciences Building & \hfill \Telefon \hspace{1 mm} 607-255-0833 \\
%Department of Physics          & \hfill \Mobilefone \hspace{1 mm} 607-220-4408 \\
%Cornell University             & \hfill \Letter \hspace{1 mm}  wt88@cornell.edu\\
%Ithaca, NY 14853 USA           & \hfill www.lepp.cornell.edu/{\raise.17ex\hbox{$\scriptstyle\sim$}}wt88 \\
%%%
1645 International Dr, Unit 114  & \hfill \Mobilefone \hspace{1 mm} 703-362-0289 \\
McLean, Virginia 22102           & \hfill \Letter \hspace{1 mm}  don.teo@gmail.com\\
USA                             %& \hfill \Letter \hspace{1 mm}  wt88@cornell.edu\\
%%%
%33 Terraview Blvd.             & \hfill \Telefon \hspace{1 mm} 416-383-1568 \\
%Toronto, ON                    & \hfill \Mobilefone \hspace{1 mm} 416-887-6881 \\
%M1R 4L8, Canada                %& \hfill \Letter \hspace{1 mm}  wt88@cornell.edu\\
%& \hfill \Letter \hspace{1 mm}  don.teo@gmail.com\\
%%%
\end{tabular}

%add objective for the financial positions
%\section{\sc Objective} To secure a position applying quantitative research and analysis skills to solve financial modelling and risk assessment problems.

%\section{\sc Research Interests}
%Bayesian statistics, spatial statistics, nonparametric regression,
%statistical methods for large datasets, statistics for public policy

\section{\sc Education}
{\bf Ph.D. Physics}, \textit{Cornell University} \hfill {\bf January, 2013}\\
%{\em Department of Statistics}
\vspace*{-.1in}
\begin{list1}
%\item[] Field: Experimental Particle Physics
\item[] Thesis: \textit{Search for Supersymmetry with b-quark Jets and Missing
\newline \hspace*{10.5 mm} Transverse Energy in pp Collisions at $\sqrt{s}=7$ TeV}
%\item[] Advisor: Peter Wittich
%\item[] Expected Date of Graduation: December 2012
%\begin{list2}
%\vspace*{.05in}
%\item Dissertation Topic:  ``Nonstationary Covariance Models for
%  Spatial Data and Regression Problems''
%\item Dissertation Topic:  ``Hierarchical Models for Multiple Ratings
%  in Performance-Based\\ \hspace*{1.23in} Student Assessments.''
%\item Advisor:  Mark J. Schervish
%\end{list2}
%\vspace*{.05in}
%\item[] M.S., Statistics,  May 2000
\end{list1}
%\vspace{-.05mm}
\vspace{-.1in}
{\bf M.Sc. Physics}, \textit{Cornell University} \hfill {\bf June, 2010}\\
%\vspace{-.1in}
{\bf B.Sc. Mathematics and Physics}, \textit{University of Toronto} \hfill {\bf June, 2007}\\
\vspace*{-.1in}
%\begin{list1}
%\item[] Graduated with High Distinction
%\end{list1}
%{\em Department of Mathematics and Statistics}
%\vspace*{-.1in}
%\begin{list1}
%\item[] B.Sc. Honours, Mathematics and Physics, 2002-2007
%\end{list1}
\vspace{-.3cm}
\section{\sc Work \\ Experience}

{\bf Opower}, Arlington, Virginia, USA

\vspace{-.3cm}
{\em Implementation Engineer III} \hfill {\bf October, 2015 - Present}\\
%Supervisor: Professor Peter Wittich\\

\vspace*{-5mm}
%\underline{Physics Analysis}
\vspace*{2mm}
\begin{list2}
\item Technical lead on enterprise-level client delivery team.
  Launched Opower's Small and Medium Business
  energy efficiency product offerings for two large U.S. utility
  companies.
\item Developed business segment classification and open hours
  prediction algorithms in Python for Opower's Machine Learning Lab.
% \item \color{red}{Heat pump prediction analysis}
\end{list2}
\vspace{-.3cm}
{\em Implementation Engineer II} \hfill {\bf October, 2014 - October, 2015}\\
%Supervisor: Professor Peter Wittich\\

\vspace*{-5mm}
%\underline{Physics Analysis}
\vspace*{2mm}
\begin{list2}
\item Client-facing engineer for utility rate-pricing data acquisition
  and customer information system migration efforts. Led the
  end-to-end implementation and operational readiness testing phases
  of Opower's Peak Time Rebates product offering.
\item Built an automated customer billing and smart meter data quality
  validation framework in Ruby.
\item Led a pilot project to allow utility companies to monitor the
  import of customer smart meter data through Opower's data warehouse.
  Developed BI reporting extracts in MySQL and Hive.

\end{list2}

\vspace{-.3cm}
{\em Implementation Engineer I} \hfill {\bf July, 2013 - October, 2014}\\
\vspace*{-5mm}
\vspace*{2mm}
\begin{list2}
\item Worked with utility companies on data integration projects for
  Opower's energy efficiency product offerings.  Primary client-facing
  engineer for two of Opower's pilot Behavioral Demand Response
  product launches. Built internal tools in Ruby, Hive, and D3.js to
  assess quality of large-scale smart meter data.
\end{list2}

%{\bf Compact Muon Solenoid (CMS) Experiment}
{\bf Cornell University}, Ithaca, New York, USA

\vspace{-.3cm}
{\em Graduate Research Assistant, Cornell University} \hfill {\bf August, 2008 - January, 2013}\\
%Supervisor: Professor Peter Wittich\\

\vspace*{-5mm}
%\underline{Physics Analysis}
\vspace*{2mm}
\begin{list2}
%For Physics people
%\item Significant contributions to the first cross-section measurement
%  of $t\bar{t}$ production at CMS in the lepton+jets channel with
%  semi-leptonically decaying $b$-quarks, including $b$-jet
%  identification optimization, data-driven $W$+jets background
%  estimation, and selection efficiency and systematic error estimates.
%\item Significant contributions to the search for Supersymmetry (SUSY)
%  with hadronic final state topologies enriched in $b$-quarks,
%  including data-driven estimates of the $t\bar{t}$ and QCD
%  backgrounds, and estimates of systematic errors on the signal
%  selection efficiency.
%\item Development of high-level trigger strategy for SUSY analysis in
%  hadronic+$b$-tag final state for instantaneous luminosities up to
%  $7\times10^{33} cm^{-2} s^{-1}$, including development of utility
%  triggers for the collection of control samples and measurement of
%  trigger efficiencies.
%\item Contributions to the study and improvement of $b$-jet
%  identification and missing transverse energy reconstruction in the
%  high-level trigger.


%%For non-physics people
%\item Significant contributions to the first measurement of the top
%  quark pair production cross section at CMS in the single-lepton
%  decay channel with semi-leptonically decaying $b$ quarks, including
%  $b$ quark identification optimization, background determination, and
%  selection efficiency/systematic error estimation.
%\item Major contributions to the search for new physics with
%  final-state topologies enriched in $b$ quarks, including data-driven
%  estimates of the main backgrounds and the determination of
%  systematic errors on the signal selection efficiency.
%\item Developed long-term analysis trigger strategy %up to the highest
%                                                    %beam luminosities
%  and designed custom utility triggers for the collection of data
%  control samples and the measurement of trigger efficiencies.
%\item Played key role in the improvement of the performance of physics
%  object reconstruction at the high-level trigger.


%For really non-physics people
\item Member of the CMS experiment at the CERN laboratory in Geneva,
  Switzerland. Developed a suite of data quality monitoring
  visualization software tools for the detector trigger systems.
  Created software to extrapolate detector trigger rates into high beam
  luminosity regimes.
\item Worked in tight-knit teams on large-scale data analysis projects
  for the measurement of the production rate of the top quark and the
  search for new-physics particles.  Analysis responsibilities
  included the processing and storing of the datasets, the
  optimization of the event selection strategy, the determination of
  efficiencies and systematic uncertainties, and the development of
  novel background-estimation methods.

\end{list2}

{\bf Cornell University}, Ithaca, New York, USA

\vspace*{-0.15in}
{\em Teaching Assistant and Grader} \hfill {\bf August, 2007 - May, 2009}\\
%{\em Teaching Assistant} \hfill {\bf Fall 2008}\\
\begin{list2}
\item Conducted weekly tutorial and laboratory sessions, prepared
  quizzes, and graded homework sets and examinations in fundamental
  physics courses for engineers and pre-med majors.  Taught and
  supervised a total of roughly 60 students per semester.
\item Graded homework sets for advanced graduate course in quantum field theory.
\end{list2}
%\underline{Detector Projects}
%\vspace*{2mm}
%\begin{list2}
%\item Developed comprehensive suite of data quality monitoring (DQM)
%  visualization tools for the Level 1 and high-level trigger systems,
%  particularly for the muon trigger systems.  The tools provide the
%  DQM shift crew real-time diagnostics on the performance of the
%  trigger systems and create alarms when unexpected trigger rates are
%  detected.
%\item Early contributions to the systematic study of the behaviour of
%  trigger rates in an environment with a large number of simultaneous
%  proton-proton collisions.
%\end{list2}

\newpage

%{\bf Tokai-to-Kamioka (T2K) Experiment}
{\bf University of Toronto / York University}, Toronto, Ontario, Canada

\vspace{-.3cm}
{\em Undergraduate Research Assistant} \hfill {\bf May, 2006 - May, 2007}\\
%Supervisor: Professor John Martin\\
\vspace*{-2mm}
\begin{list2}
\item Member of the Optical Transition Radiation proton beam
  monitor group for the Tokai-to-Kamioka neutrino oscillation
  experiment. Built simulation model of beam monitor system in
  ASAP\textsuperscript{\texttrademark} ray-tracing software framework
  and studied effects of system misalignment and light efficiency on
  beam image. Evaluated impact of beam size uncertainty on final neutrino
  measurements using Monte Carlo simulations.
%neutrino flux
%  far/near ratio with T2K simulation.
%\end{list2}

%{\em Summer Research Assistant, York University} \hfill {\bf May, 2006
%  - August, 2006}\\
%Supervisor: Professor Sampa Bhadra\\
%\vspace*{-2mm}
%\begin{list2}
\item Implemented pattern-finding and distortion correction methods
  for calibrating beam images using beam monitor system prototype and
  custom-made ray-tracing simulations.
\end{list2}

%\newpage

%{\bf Collider Detector at Fermilab (CDF) Experiment}
%
%\vspace{-.3cm}
%{\em Undergraduate Research Project, University of Toronto} \hfill {\bf January, 2006 - May, 2006}\\
%%Supervisor: Professor Pierre Savard\\
%\vspace*{-2mm}
%\begin{list2}
%%\item Studied various systematic uncertainties to the top mass
%%  measurement using the Neutrino Weighting Algorithm method in the
%%  dilepton decay channel.
%\item Studied various systematic uncertainties on the measurement of
%  the top quark mass using the Neutrino Weighting Algorithm method in
%  the dilepton decay channel.
%\end{list2}

{\bf University of Toronto}, Toronto, Ontario, Canada

\vspace{-.3cm}
{\em Summer Research Assistant} \hfill {\bf May, 2004 - August, 2004}\\
%Supervisor: Professor Aephraim Steinberg\\
\vspace*{-2mm}
\begin{list2}
\item Developed a Fabry-Perot interferometer for laser calibration in
  the Quantum Optics research group.  Repaired and improved
  functionality of laser diode modules using LabVIEW platform.
\end{list2}

%\newpage

%\section{\sc Teaching Experience}
%
%{\bf Cornell University}, Ithaca, New York, USA
%
%\vspace*{-0.15in}
%{\em Teaching Assistant and Grader} \hfill {\bf August, 2007 - May, 2009}\\
%%{\em Teaching Assistant} \hfill {\bf Fall 2008}\\
%\begin{list2}
%\item Conducted weekly tutorial and laboratory sessions, prepared
%  quizzes, and graded homework sets and examinations in fundamental
%  physics courses for engineers and pre-med majors.  Taught and
%  supervised a total of roughly 60 students per semester.
%\item Graded homework sets for advanced graduate course in quantum field theory.
%%%\item PHYS207 - Fundamentals of Physics I%, taught by Professor Matthias Liepe
%%\item Fundamentals of Physics I and II%, taught by Professor Matthias Liepe
%%%\newline
%%%{\em Teaching Assistant} \hfill {\bf Spring 2008,2009}\\
%%%\item PHYS208 - Fundamentals of Physics II%, taught by Lecturer Robert Fulbright
%%%\item Fundamentals of Physics II%, taught by Lecturer Robert Fulbright
%%%{\em Teaching Assistant} \hfill {\bf Fall 2007}\\
%%%\newline
%%%\item PHYS213 - Thermal Physics; Electricity \& Magnetism%, taught by Professor Maxim Perelstein
%%\item Thermal Physics, Electricity \& Magnetism for Engineers%, taught by Professor Maxim Perelstein
%%%\vspace*{-0.15in}
%%%{\em Grader} \hfill {\bf Fall 2007}\\
%%%\newline
%%%\item PHYS651 - Relativistic Quantum Field Theory 1%, taught by Professor Yuval Grossman
%%\item Relativistic Quantum Field Theory 1%, taught by Professor Yuval Grossman
%\end{list2}


%\vspace{-.3cm}
%{\em Graduate Student} \hfill {\bf August, 1998 - present}\\
%Includes current Ph.D.~research, Ph.D.~and Masters level coursework and
%research/consulting projects.
%
%%\vspace{-.1cm}
%{\em Instructor} \hfill {\bf May - June, 2002}\\
%Co-taught graduate level course for the Master of Science in
%Computational Finance program.  Shared responsibility for lectures, exams,
%homework assignments, and  grades.
%\vspace*{.05in}
%\begin{list2}
%\item 46-731 Probability and Statistics, Summer 2002.
%\end{list2}
%
%
%%\vspace{-.1cm}
%{\em NSF VIGRE Teaching Fellow} \hfill {\bf January - May, 2001}\\
%Head teaching assistant.
%Duties included  shared administrative responsibilities with faculty
%instructor, fielding of all student inquiries, and oversight of
%graduate student teaching assistants and graders.
%\vspace*{.05in}
%\begin{list2}
%\item 36-217 Probability Theory and Random Processes, Spring 2001.
%\end{list2}
%
%%\vspace{-.1cm}
%{\em Teaching Assistant} \hfill {\bf August, 2001  - present}\\
%Duties at various times have included
%office hours and leading weekly computer lab exercises.


\section{\sc Honours and Awards}
%\vspace*{-2mm}
%\renewcommand{\arraystretch}{2}\addtolength{\tabcolsep}{-1pt}
\renewcommand{\arraystretch}{1.5}\addtolength{\tabcolsep}{-1pt}
\begin{tabular}{@{}p{4.5in}p{1.37in}}
%\begin{list1}
%\item[] NSERC (Natural Sciences and Engineering Research Council of Canada) Postgraduate Fellowship   \hfill {\bf 2009 - 2012}
%\item[] AAPT (American Association of Physics Teachers) Outstanding Teaching Assistant of the Year    \hfill {\bf 2008}
%\item[] Samuel Beatty In-Course Award, University of Toronto  \hfill {\bf 2007}
%%Dean's List, University of Toronto & \hfill {\bf 2004, 2006, 2007}\\
%\item[] Donald G. Ivey Scholarship in Physics, University of Toronto  \hfill {\bf 2004}
%\end{list1}

%\color{red}{Top X\% in Kaggle NCAA March Madness competition} & \hfill {\bf 2016}\\
%\color{red}{Top X\% in Kaggle Yelp image classification competition} & \hfill {\bf 2016}\\
%\color{red}{Top X\% in Kaggle DrivenData competition} & \hfill {\bf 2016}\\
NSERC (Natural Sciences and Engineering Research Council of Canada) Postgraduate Fellowship &  \hfill {\bf 2009 - 2012}\\
AAPT (American Association of Physics Teachers) Outstanding Teaching Assistant of the Year  &  \hfill {\bf 2008}\\
%Samuel Beatty In-Course Award, University of Toronto & \hfill {\bf 2007}\\
%Dean's List, University of Toronto & \hfill {\bf 2004, 2006, 2007}\\
Donald G. Ivey Scholarship in Physics, University of Toronto & \hfill {\bf 2004}\\

\end{tabular}

%NSERC (Natural Sciences and Engineering Research Council of Canada) Postgraduate Fellowship, 2009-2012
%
%\vspace*{-2.5mm}
%AAPT (American Association of Physics Teachers) Outstanding Teaching Assistant of the Year, 2008
%
%\vspace*{-2.5mm}
%Samuel Beatty In-Course Award, University of Toronto, 2007
%
%\vspace*{-2.5mm}
%Donald G. Ivey Scholarship in Physics, University of Toronto, 2004


%\vspace*{-2.5mm}



\section{\sc Technical \\ Skills}
\begin{list2}
\item Analysis Tools: Significant experience with Scikit-learn,
  pandas,  Past experience with Maple, Octave, R, ROOT. %\color{red}{theano, xgboost, keras}.
%Past experience in Excel, Maple, Mathematica, NumPy, pandas, R.
%  with SAS; extensive use of C and Fortran statistical libraries.
\item Programming languages: Proficient in C++, Java, Python, Ruby.
  Experience in Perl, Javascript, Unix shell scripts.
\item Web developent: Experience in D3.js, Ruby on Rails.
\item Database systems: Significant experience with MySQL. Experience with Apache HBase and Hive.
  %, \color{red}{HBase, Elasticsearch}
\item Experience with Pentaho Data Integration (Kettle).
\item Extensive experience with Git revision control system.
\item Configuration management tools: Puppet, Chef.
%  MPI parallel processing library.
%\item Applications: Generic Mapping Tools (GMT) - Unix mapping software, \LaTeX, common Windows
%  database, spreadsheet, and presentation software
%\item Algorithms: Experience programming Markov Chain Monte Carlo
%  simulations of Bayesian posterior distributions
%\item Operating Systems:  Significant experience with Unix/Linux, Windows.
%\item Experience with batch submission systems, grid computing, cloud computing.
%\item Monte Carlo generators: Pythia6, Madgraph
\item Experience with large-scale data analysis ($>$100 TB), Monte-Carlo simulations,
and machine learning techniques.
%\item Past experience with Blackboard course managment system
%\item Machine Learning.  Completed December, 2016 at coursera.org.  Taught by Andrew Ng, Stanford University
\end{list2}



%\section{\sc Selected Publications}
%\begin{list2}
%%\item \textit{Search for supersymmetry in events with b-quark jets and
%%  missing transverse energy in pp collisions at 7 TeV,} CMS
%%  Collaboration, 2012, Submitted to PRD
%\item \textit{Search for Supersymmetry in Events with b-quark Jets and Missing Transverse Energy in pp Collisions at 7 TeV,} CMS Collaboration, 2012, Phys. Rev. D 86 072010
%\item \textit{Measurement of the $t\bar{t}$ Production Cross Section
%  in $pp$ Collisions at 7 TeV in Lepton + Jets Events Using b-quark
%  Jet Identification,} CMS Collaboration, 2011, Phys. Rev. D 84 092004
%%\item \textit{CMS Data Processing Workflows during an Extended Cosmic Ray Run,} CMS Collaboration, 2010, JINST 5 T03006
%\item \textit{Commissioning of the CMS High-Level Trigger with cosmic rays,} CMS Collaboration, 2010, JINST 5 T03005
%\end{list2}
%Paciorek, C.J., J.S. Risbey, V. Ventura, and R.D.Rosen. 2002. Multiple indices of Northern Hemisphere Cyclone
%Activity, Winters 1949-1999. Journal of Climate 15:1573-1590.
%
%Paciorek, C.J., R. Condit, S.P. Hubbell, and R.B. Foster.  2000.
%The demographics of resprouting in tree and shrub species of a moist
%tropical forest.  Journal of Ecology 88:765-777.
%
%Paciorek, C.J., B.R. Moyer, R.A. Levin, and S.L. Halpern.  1995.
%Pollen consumption by hummingbird flower mite {\it Proctolaelaps
%  kirmsei} and possible fitness effects on {\it Hamelia patens}.
%Biotropica 27:258-262.  (author order determined by lot)
%
%\section{\sc Papers in preparation}
%
%Ventura, V., C.J. Paciorek, and J.S. Risbey.  Controlling the proportion of falsely-rejected hypotheses when conducting multiple tests with geophysical data.
%
%Ickes, K., C.J. Paciorek, and S. Thomas.  Effects of wild pigs on
%forest demographic processes in Malaysia.
%
%\section{\sc Conference Presentations}
%Paciorek, C.J., J.S. Risbey, V. Ventura, and R.D.Rosen.  2001.  Changes in Northern Hemisphere winter storm activity (1949-1999) based
%on a comparison of cyclone indices.  8th International Meeting on
%Statistical Climatology, Luneberg, Germany, March, 2001.
%
%Paciorek, C.J. and R. Rosenfeld.  2000.  Minimum classification error
%training in exponential language models.  2000 Spring Transcription
%Workshop, College Park, Maryland.
%\vspace*{-.25in}
%\begin{verbatim}http://www.nist.gov/speech/publications/tw00/html/abstract.htm#cp1-50\end{verbatim}
%
%

%\section{\sc Collaboration Public Analysis Documents}
%\begin{list2}
%\item CMS Physics Analysis Summary SUS-12-003, \textit{Search for Supersymmetry in Events with b-quark Jets and Missing Transverse Energy in pp Collisions at 7 TeV.}
%%\item CMS Physics Analysis Summary BTV-11-002, \textit{Status of b-tagging tools for 2011 data analysis.}
%\item CMS Physics Analysis Summary SUS-11-006, \textit{Search for New Physics in Events with b-quark Jets and Missing Energy in Proton-Proton Colllisions at 7 TeV.}
%\item CMS Physics Analysis Summary TOP-10-004, \textit{Selection of Top-Like Events in the Dilepton and Lepton-plus-Jets Channels in Early 7 TeV Data.}
%\end{list2}


%\section{\sc Conference Talks}


%%\section{\sc Conference Talks, Workshops, and Schools}
%%%\section{\sc Conference Talks}
%%\begin{list2}
%%\item \textit{Search for new physics in events with b-jets and missing transverse energy in pp collisions at 7 TeV}, Parallel talk at APS 2012 April Meeting, 31 March to 3 April 2012, Atlanta, GA, USA
%%%\end{list2}
%%%\section{\sc Workshops and Schools}
%%%\begin{list2}
%%\item \textit{Monte Carlo Tools for Beyond the Standard Model Physics}, 22-24 March 2012, Cornell University, Ithaca, NY, USA
%%\item \textit{Excellence in Detectors and Instrumentation Technologies}, 13-24 February 2012, FNAL, Batavia, IL, USA
%%\end{list2}
%%
%%\section{\sc Languages}
%%\begin{list2}
%%\item Fluent: English, Mandarin. Intermediate: Cantonese, French.  Basic: Japanese.
%%\end{list2}


\end{resume}
\end{document}
